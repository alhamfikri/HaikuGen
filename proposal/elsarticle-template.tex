%%
%% Copyright 2007, 2008, 2009 Elsevier Ltd
%%
%% This file is part of the 'Elsarticle Bundle'.
%% ---------------------------------------------
%%
%% It may be distributed under the conditions of the LaTeX Project Public
%% License, either version 1.2 of this license or (at your option) any
%% later version.  The latest version of this license is in
%%    http://www.latex-project.org/lppl.txt
%% and version 1.2 or later is part of all distributions of LaTeX
%% version 1999/12/01 or later.
%%
%% The list of all files belonging to the 'Elsarticle Bundle' is
%% given in the file `manifest.txt'.
%%

%% Template article for Elsevier's document class `elsarticle'
%% with numbered style bibliographic references
%% SP 2008/03/01
%%
%%
%%
%% $Id: elsarticle-template-num.tex 4 2009-10-24 08:22:58Z rishi $
%%
%%
\documentclass[final,12pt,3p]{elsarticle}
\makeatletter
\def\ps@pprintTitle{%
	\let\@oddhead\@empty
	\let\@evenhead\@empty
	\def\@oddfoot{\centerline{\thepage}}%
	\let\@evenfoot\@oddfoot}
\makeatother
%% Use the option review to obtain double line spacing
%% \documentclass[preprint,review,12pt]{elsarticle}

%% Use the options 1p,twocolumn; 3p; 3p,twocolumn; 5p; or 5p,twocolumn
%% for a journal layout:
% %\documentclass[final,1p,times]{elsarticle}
%% \documentclass[final,1p,times,twocolumn]{elsarticle}
% % \documentclass[final,3p,times]{elsarticle}
%% \documentclass[final,3p,times,twocolumn]{elsarticle}
% % \documentclass[final,5p,times]{elsarticle}
% % \documentclass[final,5p,times,twocolumn]{elsarticle}

%% if you use PostScript figures in your article
%% use the graphics package for simple commands
%% \usepackage{graphics}
%% or use the graphicx package for more complicated commands
%% \usepackage{graphicx}
%% or use the epsfig package if you prefer to use the old commands
%% \usepackage{epsfig}

%% The amssymb package provides various useful mathematical symbols
\usepackage{amssymb}
%% The amsthm package provides extended theorem environments
%% \usepackage{amsthm}

%% The lineno packages adds line numbers. Start line numbering with
%% \begin{linenumbers}, end it with \end{linenumbers}. Or switch it on
%% for the whole article with \linenumbers after \end{frontmatter}.
%% \usepackage{lineno}

%% natbib.sty is loaded by default. However, natbib options can be
%% provided with \biboptions{...} command. Following options are
%% valid:

%%   round  -  round parentheses are used (default)
%%   square -  square brackets are used   [option]
%%   curly  -  curly braces are used      {option}
%%   angle  -  angle brackets are used    <option>
%%   semicolon  -  multiple citations separated by semi-colon
%%   colon  - same as semicolon, an earlier confusion
%%   comma  -  separated by comma
%%   numbers-  selects numerical citations
%%   super  -  numerical citations as superscripts
%%   sort   -  sorts multiple citations according to order in ref. list
%%   sort&compress   -  like sort, but also compresses numerical citations
%%   compress - compresses without sorting
%%
%% \biboptions{comma,round}

% \biboptions{}

\usepackage[numbers]{natbib}

\usepackage{array}
\newcolumntype{C}[1]{>{\centering\let\newline\\\arraybackslash\hspace{0pt}}m{#1}}


\bibliographystyle{unsrtnat}
\begin{document}

\begin{frontmatter}

\title{Comparative Local Search Approaches to Solve Flexible Job Shop Scheduling Problem}

\author{Alham Fikri Aji \\University of Edinburgh\\s1447728@sms.ed.ac.uk}

\begin{abstract}
Flexible job shop scheduling is one variant of job shop scheduling problem. Job scheduling problem is considered as NP-hard combinatorics problem. Therefore, approach for finding the optimal result for flexible job shop scheduling is considered ineffective. Local search as heuristic approach for finding near-optimal solution in difficult combinatorics problem is one option to approach flexible job shop scheduling problem. Three different local search techniques by using genetic algorithm, tabu search, and artificial bee colony algorithms are discussed. Those techniques are compared with benchmarked flexible job shop scheduling dataset. In terms of result, genetic algorithm and artificial bee colony outperform in most cases compared to tabu search. In term of time, genetic algorithms provide faster computational time compared to the other algorithms.  
\end{abstract}


\end{frontmatter}

%%
%% Start line numbering here if you want
%%
% \linenumbers
	
\section{Introduction}

.. 

Automatic poetry generation one challenge  in computational creativity. Computational creativity is a field to explore and utilise Artificial Intelligence (AI) to work in creative object domain such as song, performance, poetry, visual design, or other areas \cite{colton2012computational}. In such creative domain, emotional and intelligence aspect of the author often involved to achieve the 


<>




\section{Background}


\subsection{Haiku}


Haiku is traditional Japanese poetry. Rules of writing Haiku are clear in Japan. However, the problem occurs when Haiku is adopted into different language \cite{Haiku_poem}. The issue comes from different linguistic concepts and designs from Japanese to a foreign language. Some Japanese language concepts simply do not exist in English. Cultural difference is also one reason of variety rules in foreign Haiku. Therefore, there is no strict rule of writing Haiku in English \cite{Haiku_JP}.


Haiku is three-parts short poem consists of 17 syllables. The syllables are further distributed so that the first and the last part have 5 syllables and the second part has 7 syllables. However, converting this rule from the Japanese language which has monosyllabic phonetic system into English which has different phonetic system can sometimes be challenging  \cite{Haiku_poem}. Thus, 5,7,5 syllables distribution rule is still important in English Haiku, but not completely mandatory \cite{Haiku_JP}.



Romaji:

~\\
{\centering\textit{
yuku haru \textbf{ya} \\
tori naki uo no \\
me wa namida}  \\
(Basho, tr. Shirane)
\par
}


Translation:\\

{\centering

The passing of spring \textbf{--}\\
The birds weep and in the eyes\\
Of fish there are tears.\\
(Translated by \citeauthor{keene1999travelers}) \cite{keene1999travelers}
\par

}


~\\


Haiku consists of two different sub-ideas, separated by \textit{kireji}, or "cutting word". In Japanese, \textit{kireji} is used to extend emotional context. It may be used as punctuation. As \textit{kireji} does not exist in English, it is replaced by a punctuation mark, such as dash or question mark. In some cases, \textit{kireji} is simply unmarked and understood as implied delay \cite{Haiku_kireji}. In the example above, \textit{kireji} "ya" is often used to emphasises the previous word. In English, it is translated with a long dash.


There is no specific genre in Haiku. Any subjects can be written in Haiku \cite{Haiku_JP}. Haiku poem often delivers the content physically. This expression can be achieved by choosing words that relate with one of the human senses such as visual, hearing, or touch. Moreover, Haiku uses \textit{kigo}, or 'seasonal words' to refer and visualise the poem into one of the seasons in Japan. However, modern Haiku may do not contains any seasonal word. \cite{Haiku_JP}. It is also important to collaborate the five senses perception with correct seasonal theme \cite{Haiku_poet}.


\subsection{Automated Poetry Generation}


We define automatic poetry text generation simply as automatic text generation with additional poetic constraint. The . Some poetic measurements have been described and will be discussed in next part of this chapter. Automated poetry generation is not something new. Previous attempts have been done to develop poetry generator. 


\citeauthor{rashelpemuisi} develop template based poetry generation. The template is manually chosen from some famous poem by stripping out some words in certain grammar category such as noun or verb. A poem was created by filling in the stripped out words with a new one \cite{rashelpemuisi}. Our proposed approach is based on this idea. Another approach is done by applying a genetic algorithm to construct a poem\cite{manurung2012using}.


\subsection{Measuring Poetry}


As mentioned in the first chapter, one major issue of poetry generation is how to objectively evaluate the result. Hence, a method to answer "how good your poetry" should be defined formally. In this part, we some definitions and methods to achieve such goal.


\citeauthor{manurung2000towards} mentioned that there are three aspects of a poem that can be scored. The first one is the phonetics. This information can be evaluated through the poem's phonetic structure such as rhyme, metre, alliteration, etc. The second measurement is linguistic. Word and syntax choices are measured at this point. The last part of the evaluation is semantics, where a poem is scored based on its semantics structure relative to the desired semantics\cite{manurung2000towards}.


\citeauthor{colton2012full} suggest a more statistical approach to measuring a poem. Evaluation is done by analysing each word and assign some scores based on several criterions. A flamboyance is defined by word frequency, where a word obtained less score based how frequent it used in the constructed poem. It also computes the relevance of each word from given baseline article. Lastly, appropriateness is a score defined by its average word's sentiment distance towards a given sentiment level \cite{colton2012full}. 
\section{Resources}

\subsection{WordNet}

WordNet is huge English lexical database that focuses on relationships between the meaning and the concept of words. WordNet groups nouns, verb, adjectives, and adverbs based on their meaning and usage into set called sysnet. Words in a sysnet shares same concept and can be substituted by each other in most cases. WordNet has about 117.000 sysnets covering broad topics.

Relationship between sysnets are defined by linking them with conceptual relational attribute. This relational


\subsection{Standford PoS Tagger}

Standford implement


\subsection{CMU Pronouncing Dictionary}

Carnegie Mellon University Pronouncing Dictionary (CMUPD) provides machine-readable pronunciation dictionary for over 133.000 words. It uses 39 set of phonemes based on ARPAbet symbol set. Each vowel contains lexical stress pattern information, further categorised into three different levels: 0 for no stress, 1 for primary stress, and 2 for secondary stress\cite{CMUDict}. The stress pattern and pronunciation are useful to analyse the beautifulness of a poetry.

\begin{table}
	\centering
	
	\begin{tabular}{|c|c|}
		\hline  \textbf{Word} & \textbf{Pronunciation} \\ 
		\hline Haiku  & HH AY1 - K UW0\\
		\hline Poetry  & P OW1 - AH0 - T R IY0 \\
		\hline Thesis  &  TH IY1 - S AH0 S0 \\
		\hline
	\end{tabular} 
	\label{Sylla}
	\caption{Examples of Syllabified CMU Pronunciation Dictionary}
\end{table}

\citeauthor{bartlett2009syllabification} further improve the CMUPD and add additional syllables information. This syllabified CMUPD provides syllable boundaries that splits between one syllable to another \cite{bartlett2009syllabification}. This information can be used to compute the total syllables of given word which is invaluable resource for fitting the syllable constraints in Haiku. Table .. shows some examples of words and their corresponding pronunciation data.


\section{Project Specification}

In this project, we aimed to build automated Haiku generator. This Haiku generator will takes input of one or some topic keywords from user, and randomly generates Haiku that related with the keyword.This keyword is not mandatory, topic will be randomly chosen in case when user does not provide the input.

The generated Haiku will follow English Haiku rules as follow:
\begin{enumerate}
	\item It consist of three lines with 5,7,5 syllables distribution
	\item Using \textit{kireji} is not necessary
	\item Using seasonal words is preferable, but not required.
	\item Does not consider about rhymes
\end{enumerate}

However, as there is no strict rule of writing English Haiku, we allow user to customize the output Haiku structure. Therefore, we put the rules above as default rule and user may change it preferences as desired. 

This project will be implemented in Java, thus we expect that it should run in most standard computers. A Java applet of the project will be developed and further deployed on web platform. 

 
	

%% References
%%
%% Following citation commands can be used in the body text:
%% Usage of \cite is as follows:
%%   \cite{key}         ==>>  [#]
%%   \cite[chap. 2]{key} ==>> [#, chap. 2]
%%

%% References with bibTeX database:

% \bibliographystyle{elsarticle-num}
% \bibliographystyle{elsarticle-harv}
% \bibliographystyle{elsarticle-num-names}
% \bibliographystyle{model1a-num-names}
% \bibliographystyle{model1b-num-names}
% \bibliographystyle{model1c-num-names}
% \bibliographystyle{model1-num-names}
% \bibliographystyle{model2-names}
% \bibliographystyle{model3a-num-names}
% \bibliographystyle{model3-num-names}
% \bibliographystyle{model4-names}
% \bibliographystyle{model5-names}
% \bibliographystyle{model6-num-names}
~\\
\bibliography{sample}

\end{document}

%%
%% End of file `elsarticle-template-num.tex'.
