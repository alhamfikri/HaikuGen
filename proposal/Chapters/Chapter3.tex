\section{Resources}

\subsection{WordNet}

WordNet is huge English lexical database that focuses on relationships between the meaning and the concept of words. WordNet groups nouns, verb, adjectives, and adverbs based on their meaning and usage into set called sysnet. Words in a sysnet shares same concept and can be substituted by each other in most cases. WordNet has about 117.000 sysnets covering broad topics.

Relationship between sysnets are defined by linking them with conceptual relational attribute. This relational


\subsection{Standford PoS Tagger}

Standford implement


\subsection{CMU Pronouncing Dictionary}

Carnegie Mellon University Pronouncing Dictionary (CMUPD) provides machine-readable pronunciation dictionary for over 133.000 words. It uses 39 set of phonemes based on ARPAbet symbol set. Each vowel contains lexical stress pattern information, further categorised into three different levels: 0 for no stress, 1 for primary stress, and 2 for secondary stress\cite{CMUDict}. The stress pattern and pronunciation are useful to analyse the beautifulness of a poetry.

\begin{table}
	\centering
	
	\begin{tabular}{|c|c|}
		\hline  \textbf{Word} & \textbf{Pronunciation} \\ 
		\hline Haiku  & HH AY1 - K UW0\\
		\hline Poetry  & P OW1 - AH0 - T R IY0 \\
		\hline Thesis  &  TH IY1 - S AH0 S0 \\
		\hline
	\end{tabular} 
	\label{Sylla}
	\caption{Examples of Syllabified CMU Pronunciation Dictionary}
\end{table}

\citeauthor{bartlett2009syllabification} further improve the CMUPD and add additional syllables information. This syllabified CMUPD provides syllable boundaries that splits between one syllable to another \cite{bartlett2009syllabification}. This information can be used to compute the total syllables of given word which is invaluable resource for fitting the syllable constraints in Haiku. Table .. shows some examples of words and their corresponding pronunciation data.

