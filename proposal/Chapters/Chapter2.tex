\section{Background}

\subsection{Haiku}

Haiku is traditional Japanese poetry . 

Rules of writing Haiku are clear in Japan. However, problem occurs when Haiku is adopted into different language \cite{Haiku_poem}. The issue comes from different linguistic concepts and designs from Japanese to foreign language. Some Japanese language concepts simply does not exists in English. Cultural difference is also one reason of variety rules in foreign Haiku. Therefore, there is no strict rule of writing Haiku in English \cite{Haiku_JP}.

Haiku is three-parts short poem consists of 17 syllables. The syllables are further distributed into each part with ratio of 5, 7, 5. Therefore, the first and the last part have 5 syllables and the second part has 7 syllables. However, converting this rule from Japanese language which has monosyllabic phonetic system into English which has different phonetic system can sometimes be challenging  \cite{Haiku_poem}. Thus, 5,7,5 syllables distribution rule is still important in English Haiku, but not completely mandatory \cite{Haiku_JP}.


Romaji:
\\
{\centering\textit{
		yuku haru \textbf{ya} \\
		tori naki uo no \\
		me wa namida}  \\
	(Bashō, tr. Shirane)
	\par
}

Translation:\\
{\centering
	spring going\textbf{--} \\
	birds crying and tears \\
	in the eyes of fish
	\par
}

~\\

Haiku consists of two different sub-ideas, separated by \textit{kireji}, or "cutting word". In Japanese, \textit{kireji} is used to extend emotional context. It may be used as punctuation. As \textit{kireji} does not exists in English, it is replaced by punctuation mark, such as dash or question mark. In some cases, \textit{kireji} is simply unmarked and understood as implied delay \cite{Haiku_kireji}. In example above, \textit{kireji} "ya" is often used to emphasises the previous word. In English, it is translated with long dash.


There is no specific genre in Haiku. Any subjects can be written in Haiku \cite{Haiku_JP}. Haiku poem often can be  <><>. This expression can be achieved by choosing words that relate with one of human senses such as visual, hearing, or touch. Moreover, Haiku uses \textit{kigo}, or 'seasonal words' to refer and visualise the poem into one of the seasons in Japan. However, modern Haiku may do not contains any seasonal word. \cite{Haiku_JP}. It is also important to collaborate the five senses perception with correct seasonal theme \cite{Haiku_poet}.




The underlined words in

\subsection{Automated Poetry Generation}




