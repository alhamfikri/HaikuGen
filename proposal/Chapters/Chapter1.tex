\section{Introduction}


Poetry is one of natural human method to communicate by encapsulating the message in beautiful and artistic manner. William Wordsworth said that poetry is "the spontaneous overflow of powerful feelings: it takes its
origin from emotion recollected in tranquillity" \cite{wordsworth1990preface}. More non-philosophic and restricted description defined by \citeauthor{manurung2004evolutionary}, poem is a text that is meaningful, grammatically correct, and poetic \cite{manurung2004evolutionary}.

Automatic poetry generation one challenge in artificial intelligence and computational creativity. It is considerably interesting as it works on human creativity, emotional and intelligence domain \cite{colton2012computational}. The difficulties of this task comes in various factors. One major issue lies in the poem evaluation. Evaluating poems and creative arts tends to be subjective, thus automated poetry generation faces difficulty to obtain objective result \cite{binsted1996machine}. Another challenge in poetry generation is massive requirement of the resources. In order to create poem that satisfies diverse constraints such as phonetic, syntax, grammar or semantics requires wide array of resources \cite{manurung2000towards}. 

The objective of this work is to explore further about poetry generation. This work tries to answer the challenge in computational creativity on automatically builds poetry as one of human artifact. This work will be focused into one specific genre of poem: Haiku. Although specialized, we are hoping that this work can be generalized in some aspects for broader and more generic poetry generation. 

Lastly, poetry generation could be used in various creativity related domain, such as helping tool for song composers, video game company, greeting cards industry, or be used by poet themselves. It also could be used to enhance computer to human interaction system. Personal assistant systems could embed poetry generation to make it capable to talk in poetic manner, thus making it more friendly towards the user. 

 The papers are organized as follows. The section 1 delivers the motivation and the objective of research in automated poem generation. The section 2 of this paper explains necessary background information related to this topic. This section also covers recent related works. The section 3 lists some useful resources to aid the research. The section 4 states the specification of this project. The section 5 describes detailed information of the proposed methodology. Finally, the section 6 is the last section that discuss about evaluation method that will be used in this research.



