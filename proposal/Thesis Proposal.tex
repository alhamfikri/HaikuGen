%%
%% Copyright 2007, 2008, 2009 Elsevier Ltd
%%
%% This file is part of the 'Elsarticle Bundle'.
%% ---------------------------------------------
%%
%% It may be distributed under the conditions of the LaTeX Project Public
%% License, either version 1.2 of this license or (at your option) any
%% later version.  The latest version of this license is in
%%    http://www.latex-project.org/lppl.txt
%% and version 1.2 or later is part of all distributions of LaTeX
%% version 1999/12/01 or later.
%%
%% The list of all files belonging to the 'Elsarticle Bundle' is
%% given in the file `manifest.txt'.
%%

%% Template article for Elsevier's document class `elsarticle'
%% with numbered style bibliographic references
%% SP 2008/03/01
%%
%%
%%
%% $Id: elsarticle-template-num.tex 4 2009-10-24 08:22:58Z rishi $
%%
%%
\documentclass[final,12pt,3p]{elsarticle}
\makeatletter
\def\ps@pprintTitle{%
	\let\@oddhead\@empty
	\let\@evenhead\@empty
	\def\@oddfoot{\centerline{\thepage}}%
	\let\@evenfoot\@oddfoot}
\makeatother
%% Use the option review to obtain double line spacing
%% \documentclass[preprint,review,12pt]{elsarticle}

%% Use the options 1p,twocolumn; 3p; 3p,twocolumn; 5p; or 5p,twocolumn
%% for a journal layout:
% %\documentclass[final,1p,times]{elsarticle}
%% \documentclass[final,1p,times,twocolumn]{elsarticle}
% % \documentclass[final,3p,times]{elsarticle}
%% \documentclass[final,3p,times,twocolumn]{elsarticle}
% % \documentclass[final,5p,times]{elsarticle}
% % \documentclass[final,5p,times,twocolumn]{elsarticle}

%% if you use PostScript figures in your article
%% use the graphics package for simple commands
%% \usepackage{graphics}
%% or use the graphicx package for more complicated commands
%% \usepackage{graphicx}
%% or use the epsfig package if you prefer to use the old commands
%% \usepackage{epsfig}

%% The amssymb package provides various useful mathematical symbols
\usepackage{amssymb}
%% The amsthm package provides extended theorem environments
%% \usepackage{amsthm}

%% The lineno packages adds line numbers. Start line numbering with
%% \begin{linenumbers}, end it with \end{linenumbers}. Or switch it on
%% for the whole article with \linenumbers after \end{frontmatter}.
%% \usepackage{lineno}

%% natbib.sty is loaded by default. However, natbib options can be
%% provided with \biboptions{...} command. Following options are
%% valid:

%%   round  -  round parentheses are used (default)
%%   square -  square brackets are used   [option]
%%   curly  -  curly braces are used      {option}
%%   angle  -  angle brackets are used    <option>
%%   semicolon  -  multiple citations separated by semi-colon
%%   colon  - same as semicolon, an earlier confusion
%%   comma  -  separated by comma
%%   numbers-  selects numerical citations
%%   super  -  numerical citations as superscripts
%%   sort   -  sorts multiple citations according to order in ref. list
%%   sort&compress   -  like sort, but also compresses numerical citations
%%   compress - compresses without sorting
%%
%% \biboptions{comma,round}

% \biboptions{}

\usepackage[numbers]{natbib}

\usepackage{array}
\newcolumntype{C}[1]{>{\centering\let\newline\\\arraybackslash\hspace{0pt}}m{#1}}


\bibliographystyle{unsrtnat}
\begin{document}

\begin{frontmatter}

\title{An Automated Haiku Generator}

\author{Alham Fikri Aji \\University of Edinburgh\\s1447728@sms.ed.ac.uk}

\begin{abstract}
Automated poetry generation is a challenge in artificial intelligence and computational creativity research. It covers various language aspects such as phonetic, semantic ,lexical or syntax, and is usually requires vast amount of database as basis knowledge. To answer this challenge, we propose an automated haiku generation that learns the grammatical rules from existing haiku and construct a new haiku based on some corpus. This system is able to take some keywords from users and randomly create a haiku. To evaluate the system, we propose Turing test to evaluate the similarity of our generated haiku with real handcrafted haiku. 
\end{abstract}


\end{frontmatter}

%%
%% Start line numbering here if you want
%%
% \linenumbers
	
\section{Introduction}


Poetry is one of natural human method to communicate by encapsulating the message in beautiful and artistic manner. William Wordsworth said that poetry is "the spontaneous overflow of powerful feelings: it takes its
origin from emotion recollected in tranquillity" \cite{wordsworth1990preface}. More non-philosophic and restricted description defined by \citeauthor{manurung2004evolutionary}, poem is a text that is meaningful, grammatically correct, and poetic \cite{manurung2004evolutionary}.

Automatic poetry generation one challenge in artificial intelligence and computational creativity. It is considerably interesting as it works on human creativity, emotional and intelligence domain \cite{colton2012computational}. The difficulties of this task comes in various factors. One major issue lies in the poem evaluation. Evaluating poems and creative arts tends to be subjective, thus automated poetry generation faces difficulty to obtain objective result \cite{binsted1996machine}. Another challenge in poetry generation is massive requirement of the resources. In order to create poem that satisfies diverse constraints such as phonetic, syntax, grammar or semantics requires wide array of resources \cite{manurung2000towards}. 

The objective of this work is to explore further about poetry generation. This work tries to answer the challenge in computational creativity on automatically builds poetry as one of human artifact. This work will be focused into one specific genre of poem: Haiku. Although specialized, we are hoping that this work can be generalized in some aspects for broader and more generic poetry generation. 

Lastly, poetry generation could be used in various creativity related domain, such as helping tool for song composers, video game company, greeting cards industry, or be used by poet themselves. It also could be used to enhance computer to human interaction system. Personal assistant systems could embed poetry generation to make it capable to talk in poetic manner, thus making it more friendly towards the user. 

 The papers are organized as follows. The section 1 delivers the motivation and the objective of research in automated poem generation. The section 2 of this paper explains necessary background information related to this topic. This section also covers recent related works. The section 3 lists some useful resources to aid the research. The section 4 states the specification of this project. The section 5 describes detailed information of the proposed methodology. Finally, the section 6 is the last section that discuss about evaluation method that will be used in this research.




\section{Background}

\subsection{Haiku}

Haiku is traditional Japanese poetry . 

Rules of writing Haiku are clear in Japan. However, problem occurs when Haiku is adopted into different language \cite{Haiku_poem}. The issue comes from different linguistic concepts and designs from Japanese to foreign language. Some Japanese language concepts simply does not exists in English. Cultural difference is also one reason of variety rules in foreign Haiku. Therefore, there is no strict rule of writing Haiku in English \cite{Haiku_JP}.

Haiku is three-parts short poem consists of 17 syllables. The syllables are further distributed into each part with ratio of 5, 7, 5. Therefore, the first and the last part have 5 syllables and the second part has 7 syllables. However, converting this rule from Japanese language which has monosyllabic phonetic system into English which has different phonetic system can sometimes be challenging  \cite{Haiku_poem}. Thus, 5,7,5 syllables distribution rule is still important in English Haiku, but not completely mandatory \cite{Haiku_JP}.


Romaji:
\\
{\centering\textit{
		yuku haru \textbf{ya} \\
		tori naki uo no \\
		me wa namida}  \\
	(Bashō, tr. Shirane)
	\par
}

Translation:\\
{\centering
	spring going\textbf{--} \\
	birds crying and tears \\
	in the eyes of fish
	\par
}

~\\

Haiku consists of two different sub-ideas, separated by \textit{kireji}, or "cutting word". In Japanese, \textit{kireji} is used to extend emotional context. It may be used as punctuation. As \textit{kireji} does not exists in English, it is replaced by punctuation mark, such as dash or question mark. In some cases, \textit{kireji} is simply unmarked and understood as implied delay \cite{Haiku_kireji}. In example above, \textit{kireji} "ya" is often used to emphasises the previous word. In English, it is translated with long dash.


There is no specific genre in Haiku. Any subjects can be written in Haiku \cite{Haiku_JP}. Haiku poem often can be  <><>. This expression can be achieved by choosing words that relate with one of human senses such as visual, hearing, or touch. Moreover, Haiku uses \textit{kigo}, or 'seasonal words' to refer and visualise the poem into one of the seasons in Japan. However, modern Haiku may do not contains any seasonal word. \cite{Haiku_JP}. It is also important to collaborate the five senses perception with correct seasonal theme \cite{Haiku_poet}.




The underlined words in

\subsection{Automated Poetry Generation}




 
\section{Resources}

\subsection{WordNet}

WordNet is huge English lexical database that focuses on relationships between the meaning and the concept of words \cite{wordnet1}\cite{wordnet2}. WordNet groups nouns, verb, adjectives, and adverbs based on their meaning and usage into set called sysnet. Words in a sysnet shares same concept and can be substituted by each other in most cases. WordNet has about 117.000 sysnets covering broad topics\cite{wordnet3}.

Relationship between sysnets are defined by linking them with conceptual relational attribute. Some sysnet relations examples in the wordNet are:
\begin{itemize}
	\item Hypernym
	
	Relates a concept into its superordinate concepts. In verb, it connects the verb into the superordinate events. For example: book $\rightarrow$ publication, write $\rightarrow$ create.
	\item Hyponym
	
	Simply the inversion of hypernym. It relates the connects subtype concept relations. For example: book $\rightarrow$ catalogue, textbook, story book, etc. 
	
	\item Troponym
	
	Similar with hyponym. Troponym is a verb relationship that captures subordinate event of another event. For example: travel $\rightarrow$ fly, ride, etc
	
	\item Antonym
	
	Relates the opposite of word senses. For example: happy $\rightarrow$ infelicitous, travel $\rightarrow$ stay in place.
	
	\item Derivationally Related Form
	
	Connects lemma with the same root. For example: burn $\rightarrow$ fire.
	
	\item Entailment 
	
	A verb to verb relations where one verb is involved with the other. For example: snore $\rightarrow$ sleep
	
\end{itemize}

WordNet is useful resource for text generation. It can be used to paraphrase generated text. Its relational concept is helpful to build the knowledge base of the system. WordNet can be used to enrich the system's dictionary by replacing some words with other words that shares similar usage and characteristics.

\subsection{Standford PoS Tagger}

Standford Part-of-Speech (PoS) tagger is an application with purpose to assigns part of speech label verb, noun, adjective, etc. to each input text word \cite{toutanova2003feature}. For English language, Stanford PoS tagger uses labels in Penn Treebank tag set. This tag set has a total of 36 different labels with more specified and extended labels such as verb-past, verb-present, adjective-superlative, etc\cite{marcus1993building}.

\begin{figure}[h]
	\centering
	\begin{tabular}{c}
		\textbf{Input} \\
		 \textit{A frog jumps into the pond }\\
		 
	\end{tabular} 
	\\~
	$\downarrow$
	~
	\\
	\begin{tabular}{c}
	\textbf{Output} \\
	 A/DT frog/NN jumps/VBZ into/IN the/DT pond/NN \\
	 
	\end{tabular} 
	
	\caption{Example of PoS tagger}
	\label{PoSexample}
\end{figure}

Figure \ref{PoSexample} illustrates an use example of Standford PoS tagger. An input sentence is tokenized and given a label for each token. The output of PoS tagger will be list of paired token and its speech tag.

\subsection{CMU Pronouncing Dictionary}

Carnegie Mellon University Pronouncing Dictionary (CMUPD) provides machine-readable pronunciation dictionary for over 133.000 words. It uses 39 set of phonemes based on ARPAbet symbol set. Each vowel contains lexical stress pattern information, further categorized into three different levels: 0 for no stress, 1 for primary stress, and 2 for secondary stress\cite{CMUDict}. The stress pattern and pronunciation are invaluable resources that can be used to analyze the beautifulness of a poetry.

\begin{table}[h]
	\centering
	
	\begin{tabular}{|c|c|}
		\hline  \textbf{Word} & \textbf{Pronunciation} \\ 
		\hline Haiku  & HH AY1 - K UW0\\
		\hline Poetry  & P OW1 - AH0 - T R IY0 \\
		\hline Thesis  &  TH IY1 - S AH0 S0 \\
		\hline
	\end{tabular} 
	
	\caption{Examples of Syllabified CMU Pronunciation Dictionary}
	\label{Sylla}
\end{table}

\citeauthor{bartlett2009syllabification} further improve the CMUPD and add additional syllables information. This syllabified CMUPD provides syllable boundaries that splits between one syllable to another \cite{bartlett2009syllabification}. This information can be used to compute the total syllables of given word which is invaluable resource for fitting the syllable constraints in Haiku. Table \ref{Sylla} shows some examples of words and their corresponding pronunciation data.




\section{Project Specification}


In this project, we aimed to build automated Haiku generator. This Haiku generator will take an input of one or some topic keywords from a user, and randomly generates Haiku that related to the keyword.This keyword is not mandatory, the topic will be randomly chosen in a case when a user does not provide the input.


The generated Haiku will follow English Haiku rules as follow:

\begin{enumerate}

\item It consists of three lines with 5,7,5 syllables distribution

\item Using \textit{kireji} is not necessary

\item Using seasonal words is preferable, but not required.

\item Does not consider about rhymes

\end{enumerate}


However, as there is no strict rule of writing English Haiku, we allow a user to customise the output Haiku structure. Therefore, we put the rules above as default rule and user may change it preferences as desired. 

This project will be implemented in Java, thus we expect that it should run on most standard computers. Moreover, Java is operating system independent, hence the application will run on most operating systems. To further enhance the usability, a web-based application will be deployed with a HTML5 client. This application allows people to use Haiku generator without running and installing the Java application, which should be more convenience for people with little technical and computer skill. 
\section{Methodology}

The brief concept of implementing Haiku generator is by learning the grammatical pattern and rules of existing Haiku poems and build a new grammar skeleton based on it. The generator will select words to fill-in the grammar skeleton based on statistical approach. Other resources such as CMUPD and WordNet will be supporting the decision making of words selection.

\subsection{PoS Tag Extraction}

In this phase we apply PoS tagger algorithm for every collected Haiku poems. Therefore, in this phase we aim to acquire list of grammar tag pattern. We are only interested in the grammar tag, therefore we can remove the Haiku after tagging. One simple illustration of this process is provided in table \ref{exmpl}:

\begin{table}[h]
	\centering
	\begin{tabular}{|c|c|}
		\hline Haiku & Extracted PoS Tag Information \\
		\hline Three strokes of lightning & CD NNS IN NN \\ 
		One hit mountain frightening &  CD NN NN JJ  \\ 
		Dark clouds thunder loud &  JJ NNS NN RB \\ 
		\hline
	\end{tabular} 
	\caption{Example of extracted PoS tag information of the Haiku}
	\label{exmpl}
\end{table}
	
The reason of choosing this method over using grammar tree or context free grammar is because sometimes poetry does not follow standard grammar rules. By using formal grammar rule, we might achieve less poetic result. By learning from existing Haiku, we hope that we are able to capture the grammatical pattern used in Haiku.

\subsection{Creating Haiku: Grammar Skeleton}

Grammar skeleton is sequence of tags that will be used to create a Haiku. The generated Haiku should follow the grammar tag defined by the skeleton. In this phase, grammar skeleton is created by combining several tag patterns in previous part.



\begin{table}[h]
	\centering
	\begin{tabular}{|c|c|c|}
		\hline Pattern 1 & Pattern 2 & Pattern 3 \\
		\hline CD NNS IN NN & DT NNS IN NNP &\textbf{ JJ JJ NN} \\ 
		 \textbf{CD NN NN JJ} & JJ NN CC JJ NNS & VBG IN DT NN IN NN \\ 
		 JJ NNS NN RB & \textbf{VBG DT NN} & NN IN\\ 
		\hline
	\end{tabular} 
	\\
	$\downarrow$
	\\
	\begin{tabular}{|c|}
			\hline Generated Skeleton \\
			\hline  
			 JJ JJ NN  \\ 
			  JJ NNS NN RB \\ 
			  VBG DT NN \\
			\hline
		\end{tabular} 
\end{table}

Illustration of creating a grammar skeleton based on tag data is shown above. Some basis grammar patterns are randomly chosen as parents. The grammar skeleton is generated by applying crossover from the parents. In this example, each parent inherits one line of grammar pattern. In the implementation, the rule is not strictly forced the generation made by three parents. Therefore more or less parents is possible.

\subsection{Creating Haiku: Word Filling}

Haiku will be constructed with previously defined grammar skeleton as its basis. For each speech tag, we look up word with exactly same label and fill in the slot with the chosen word. Our word resource comes from various corpus. Although some corpus such as Brown corpus has already annotated with speech tagging, we need to apply identical PoS tagger algorithm that used in constructing the skeleton for speech label consistency. To enrich our dictionary, WordNet will be used to paraphrase and replace some words in corpus with different but related terms. 

In some situation user will provide the system with some keywords. In this case, the keywords will be tagged as well before further processing. Our tagged keyword will be placed in the skeleton first, before applying general word filling. Again, WordNet can be used to transform user keywords into another similar term.

Word filling is not randomly choose any words with required speech tag. It follows haiku rule constraints and meaning constraints. The word must be chosen in particular so that it satisfy the syllables rule. In case of strict use of \textit{kiregi} and/or \textit{kigo}, those words must exists in generated haiku. Meaning constrains is on how to chose correlated words and have actual meaning. We want to make the resource usage remains minimum, therefore we choose a word if it is statistically correlated with one of some previously placed words.


\subsection{Creating Haiku: Scoring}

In this part, generated haiku will be scored to determine its quality. We use the restricted definition of poetry by \citeauthor{manurung2004evolutionary} where a poem should be meaningful, grammatically correct, and poetic. As the haiku is generated based on previously defined grammar, we can remove the second factor as a score. Therefore, our scoring consists of two parts: Meaningfulness and beautifulness. 

Meaningfulness score $M$ is calculated by statistical approach. The haiku will be scored based on statistical relationship of each word, and word to word relationship. Beautifulness score $B$ is defined by some phonetic features that could be extracted from the haiku such as rhymes, stress pattern, etc.

\begin{table}[h]
	\centering
	\begin{tabular}{c}

		$S = w_mM + w_bB$
		\\
	\end{tabular} 
	
\end{table}


The score will be a weighted sum of both components, as defined in formula above. Therefore, higher $w_m$ leads to more 'meaningful' Haiku and higher $w_b$ leads to more 'poetic' Haiku. The composition of the weights is another important research part. The challenge is how to determine the weight so that the generate haiku is not too statistically meaningful so that it only produces common sentence, but not too 'creative' so that it does not make sense at all. 

This score can be used as quality threshold to make sure our generated haiku result good. The system will reject any generated haiku lower than some defined threshold and keep generating until its score requirement is satisfied. 
\section{Evaluation}


We propose a Turing test to evaluate our system and the research outcome in general. In this test, we create several automated generated Haiku and several human created Haiku. Those Haiku will be mixed into a single set. Some respondents will then decide which one is computer generated and which one is created by real human. This test will observe how good our Haiku in terms of its similarity to real human creation. Another evaluation can be achieved by asking the respondents some question and score on various aspects of the Haiku, such as "Do you understand this Haiku?" or "How beautiful this Haiku". 


In engineering aspect, we could perform some test on how fast the system builds the Haiku. Some further observation regarding the relation between score weighting and threshold towards the execution time can be performed to find out the best configuration so that the resulting Haiku is not rubbish, yet still takes reasonable amount of computational time. Another aspect that can be observed is the correlation between the resource used towards the generated Haiku. In this evaluation, we want to find out what is the best text resources for Haiku generation and automated poetry in general. 
	

%% References
%%
%% Following citation commands can be used in the body text:
%% Usage of \cite is as follows:
%%   \cite{key}         ==>>  [#]
%%   \cite[chap. 2]{key} ==>> [#, chap. 2]
%%

%% References with bibTeX database:

% \bibliographystyle{elsarticle-num}
% \bibliographystyle{elsarticle-harv}
% \bibliographystyle{elsarticle-num-names}
% \bibliographystyle{model1a-num-names}
% \bibliographystyle{model1b-num-names}
% \bibliographystyle{model1c-num-names}
% \bibliographystyle{model1-num-names}
% \bibliographystyle{model2-names}
% \bibliographystyle{model3a-num-names}
% \bibliographystyle{model3-num-names}
% \bibliographystyle{model4-names}
% \bibliographystyle{model5-names}
% \bibliographystyle{model6-num-names}
~\\
\bibliography{sample}

\end{document}

%%
%% End of file `elsarticle-template-num.tex'.
